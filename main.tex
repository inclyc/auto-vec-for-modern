\documentclass[aspectratio=169]{ctexbeamer}
\usepackage{booktabs}
\usepackage{listings}
\usepackage{xcolor}
\usepackage{float}
\usepackage{makecell}
\usetheme{Madrid}

\input{color_style.tex}
% Add "\usepackage{xcolor}"
% define some basic colors
\definecolor{mauve}{rgb}{0.58,0,0.82}

\definecolor{codegreen}{rgb}{0,0.6,0}
\definecolor{codegray}{rgb}{0.5,0.5,0.5}
\definecolor{codepurple}{rgb}{0.58,0,0.82}
\definecolor{backcolour}{rgb}{1,1,1}


\setmonofont{inconsolata}

\lstset{
% listings sonderzeichen (for german weirdness)
literate={ö}{{\"o}}1
{ä}{{\"a}}1
{ü}{{\"u}}1,
basicstyle=\tiny\ttfamily,                    % very small code
breaklines=true,                              % break long lines
commentstyle=\itshape\color{green!50!black},  % comments are green
keywordstyle=[1]\color{blue!80!black},        % instructions are blue
keywordstyle=[2]\color{orange!80!black},      % sections/other directives are orange
keywordstyle=[3]\color{red!50!black},         % registers are red
stringstyle=\color{mauve},                    % strings are from the telekom
identifierstyle=\color{teal},                 % user declared addresses are teal
frame=l,                                      % black line on the left side of code
language=C++,                   % all code is RISC-V
tabsize=4,                                    % indent tabs with 4 spaces
showstringspaces=false                        % do not replace spaces with weird underlines
}

\lstdefinestyle{mystyle}
{
    backgroundcolor=\color{backcolour},
    commentstyle=\color{codegreen},
    keywordstyle=\color{magenta},
    numberstyle=\tiny\color{codegray},
    stringstyle=\color{codepurple},
    basicstyle=\ttfamily\footnotesize,
    breakatwhitespace=false,
    breaklines=true,
    captionpos=b,
    keepspaces=true,
    numbers=none,
    numbersep=5pt,
    showspaces=false,
    showstringspaces=false,
    showtabs=false,
    tabsize=2,
    frame=none
}

\definecolor{orange3}{rgb}{0.425,0.191,0.000}
\definecolor{scarletred3}{rgb}{0.643,0.000,0.000}
\definecolor{green3}{rgb}{0.000,0.405,0.000}
\definecolor{blue3}{rgb}{0.000,0.000,0.704}
\definecolor{aluminium1}{rgb}{0.933,0.933,0.925}
\definecolor{aluminium2}{rgb}{0.827,0.843,0.812}
\definecolor{aluminium3}{rgb}{0.729,0.741,0.714}
\definecolor{aluminium4}{rgb}{0.533,0.541,0.522}
\definecolor{aluminium5}{rgb}{0.333,0.341,0.325}
\definecolor{aluminium6}{rgb}{0.180,0.204,0.212}

\lstdefinelanguage{llvm}{
    % see https://tex.stackexchange.com/questions/137237/listings-text-highlighting-based-on-prefix
    %moredelim=[s][\color{orange3}]{x}{>},
    alsoletter={\%,\#,!},
    basicstyle=\tiny\ttfamily,
    keywordsprefix={\%},
    morekeywords={\%},
    keywordstyle=\color{orange3},
    commentstyle=\color{aluminium4},
    % this allows to color inside <memrefxf32> (i.e., f32).
    otherkeywords={index, f32, i32, f64, i8},
    keywords=[3]{index, f32, memref, i32, f64, i8, affine_map, affine_set, iter_args},
    keywordstyle=[3]\color{scarletred3},
    keywords=[4]{affine},
    keywordstyle=[4]\color{aluminium6},
    showstringspaces=false,
    breaklines=true,
    breakatwhitespace=true,
    morestring=[b]",
    stringstyle=\color{green3},
    moredelim=[s][\color{blue3}]{\#}{<},
    moredelim=[s][\color{scarletred3}]{!}{\ },
    morecomment=[l]{//},
}

\lstset{style=mystyle,language=C++}

\input{section_pages.tex}

\title{Auto Vectorization Methods \& Implementations for Modern Architecture}
\subtitle{现代体系结构自动向量化的方法与实现}
\author{龙英池}
\institute{Interns @ ISCAS}

\begin{document}
\maketitle
\section{Introduction}
\subsection{Challenges}
\begin{frame}
    \frametitle{Challenges - 并行的多样性}

    并行是目前主要算力,但是 ISA 定义多样。我们难以找到一个合适的抽象来用统一的范式进行自动向量化。

    \begin{table}
        \centering
        \caption{不同运算设备并行设计与实现}
        \scriptsize
        \begin{tabular}{ccc}
            \toprule
            设备 / ISA         & 对应的并行实现                             & 备注           \\
            \midrule
            NVIDIA/AMD GPU   & SIMT                                & 结合 SIMD 和多线程 \\
            寒武纪              & SIMD 长度可变                           & repeat 操作    \\
            升腾               & 默认操作 tensor                         & 看起来可能是多维的    \\
            以 x86 为代表的传统 CPU & 定长,通常可以同时操作标量                       & 需要对齐、尾循环处理等  \\
            RISC-V           & scatter \& gather, \& 长度可变, \& mask & V 扩展         \\
            ARM-SVE          & 类似 RVV                              & 没有 RVV 灵活    \\
            \bottomrule
        \end{tabular}
    \end{table}

    编译器难以使用统一的范式进行自动向量化。

\end{frame}

\subsection{Prior Art}

\begin{frame}
    \frametitle{Prior Art - MLIR \& LLVM}

    \begin{enumerate}
        \item MLIR - 目前没有自动向量化 \\
              Vector Dialect,面向各种 DSL 提供 n-D vector type,并可以在 lowering 到 LLVM IR 的过程中对 n-D vector 进行 re-layout,例如组成float4,
        \item LLVM - 有向量化,但主要面向传统 CPU \\
              从引入 Vector Predication 开始,主要被作为 CodeGen 的最后一关对接 MLIR,LLVM 本身的自动向量化有较大历史包袱。
    \end{enumerate}
\end{frame}

\subsection{Goal}

\begin{frame}
    \frametitle{Goal - Auto-Vectorization Service}

    \begin{enumerate}
        \item Extensible: 足够方便扩展
        \item Composible: 可以组合各种情况
        \item Simple: IR 层面尽量显然,避免多余静态分析
    \end{enumerate}
\end{frame}

\section{Virtual Vector - Vector Dialect}

\begin{frame}
    \frametitle{Virtual Vector}

    MLIR Vector Dialect 提供了对各种向量的平台无关的丰富抽象.

    作为核心数据类型的 vector 可以是:
    \begin{itemize}
        \item 多维的, 或不确定维数的
        \item 定长的, 或不确定长度的, 或变长的
        \item 可以包含几乎任意的基本数据类型, 甚至诸如 int4 等类型
    \end{itemize}

    能表示 vector 在各种层次上的操作:
    \begin{itemize}
        \item 基本的创建, 运算, 内存读写, 元素读写
        \item 贴合 LLVM intrinsic 级别的操作 (e.g. 各种底层 load/store, 一维向量操作)
        \item 一般向量操作 (e.g. reduce)
        \item 贴合 Linalg 级别的操作 (e.g. transfer\_read, transfer\_write, contract)
    \end{itemize}
\end{frame}

\begin{frame}[fragile]
    \frametitle{HPC \& nD vector}
    MLIR Vector Dialect 提供了一个 nD vector 的抽象。这种抽象对于现代 HPC 来说是必要的。
    一个简单的例子:

    \begin{lstlisting}
kernel (float4 A).     load A[0-3]; add[0-3]
kernel (float A).     load A[0-15]; add[0-3]; add[4-7]; add[8-11]; add[12-15]

vector A[25 x float4] => A[25 x 4 x float]
    \end{lstlisting}
\end{frame}



\begin{frame}
    \frametitle{Position}
    \begin{minipage}[t]{0.47\textwidth}
        \small
        \begin{itemize}
            \item \texttt{affine/scf} 等上层 Dialect 解决循环等结构操作
            \item \texttt{arith} 等基本 Dialect 完成基本的向量创建和四则运算
            \item 各种 ISA-spec Dialect 补充自己没有的底层细节
            \item \texttt{vector} 则集中表示 ``通用'' 的向量操作
        \end{itemize}
    \end{minipage}%
    % \hfill%
    \begin{minipage}[t]{0.5\textwidth}
        \begin{figure}
            \centering
            \includegraphics[width=1.0\linewidth]{images/position.png}
            \caption{Positioning in the Codegen Infrastructure}
        \end{figure}
    \end{minipage}
\end{frame}

\begin{frame}
    \frametitle{Virtual Vector Lowering Path}
    \noindent
    \begin{minipage}[t]{0.50\linewidth}
        \begin{figure}
            \centering
            \includegraphics[width=1.0\linewidth]{images/position2.png}
            \caption{MLIR Patterns}
        \end{figure}
    \end{minipage}%
    \hfill%
    \begin{minipage}[t]{0.45\linewidth}
        主要的三层中间表达:
        \begin{enumerate}
            \item Virtual Vector:使用nD vector type,\texttt{vector<4x8x128xf32>}
            \item Hardware Vector:Op $\approx$ 指令
            \item LLVM:instructions,intrinsics, $\dots$
        \end{enumerate}

        \begin{itemize}
            \item VV同层转换:例如unroll,permute
            \item VV $\rightarrow$ HWV,一些手写的pattern:例如CPU的VL;这层lower之后,一些循环优化被禁止(需要确认)
            \item HWV $\rightarrow$ LLVM,手工的
        \end{itemize}
    \end{minipage}
\end{frame}

\begin{frame}
    \frametitle{Language Lowering Path}
    \begin{figure}
        \centering
        \includegraphics[width=0.35\linewidth]{images/auto-vec-path.png}
        \caption{Lowering path from language to ISA}
    \end{figure}
\end{frame}

\begin{frame}
    \frametitle{Auto-Vectorization $\Rightarrow$ Pattern-Rewrite?}

    总体思路:自动向量化 $\rightarrow$ 将标量提升为 Vector 类型 $\rightarrow$ Aseembly

    \hspace{2em}

    如果源代码:
    \begin{enumerate}
        \item 矩阵乘法等高维形式: DSL $\rightarrow$ \texttt{linalg} ($\rightarrow$ \texttt{affine} $\rightarrow$) $\rightarrow$ \texttt{vector}
        \item 标量循环: \texttt{scf.for}/通用语言 $\rightarrow$ \texttt{affine.for} $\rightarrow$ \texttt{vector}
    \end{enumerate}

    \begin{center}
        \vspace{2em}
        挑战:如何将标量循环转换为 \texttt{vector} ?
    \end{center}

\end{frame}

\begin{frame}
    \frametitle{标量循环 $\rightarrow$ \texttt{vector} Dialect}

    \begin{center}
        挑战:如何将标量循环转换为 \texttt{vector} ?

        \vspace{1.5em}
    \end{center}

    \begin{enumerate}
        \item 从标量循环/scf/语言分析数据访问和依赖
        \item 尽可能识别特定模式并生成 \texttt{affine.for}
        \item 在 \texttt{affine} 上做嵌套循环相关的优化(例如 loop fusion, unroll 等)
        \item 再在 \texttt{affine} 上做向量化 (例如 \texttt{SuperVectorize.cpp})
    \end{enumerate}
\end{frame}

\begin{frame}[fragile]
    \frametitle{C for-loop $\rightarrow$ basic \texttt{affine} Dialect}
    \noindent
    \begin{minipage}[t]{0.3\linewidth}
        \lstinputlisting[language=C,basicstyle=\Tiny]{assets/example.c}
    \end{minipage}%
    \hfill%
    \begin{minipage}[t]{0.6\linewidth}
        \lstinputlisting[language=llvm,basicstyle=\Tiny]{assets/example_direct_affine.mlir}
    \end{minipage}
\end{frame}

\begin{frame}[fragile]
    \frametitle{Loop Normalization on \texttt{affine} Dialect}

    \begin{minipage}[t]{0.6\linewidth}
        \lstinputlisting[language=llvm]{assets/example_normalized.mlir}
    \end{minipage}%
    \hfill%
    \begin{minipage}[t]{0.4\linewidth}
        \vspace{3em}
        除了循环正规化, 在 \texttt{scf} 和 \texttt{affine} dialect 上有一些现存的循环变换 pass, 可以通过 \texttt{mlir-opt --help} 查看.
    \end{minipage}

\end{frame}

\begin{frame}[fragile]
    \frametitle{Induction Variable Elimination on \texttt{affine} Dialect}

    \lstinputlisting[language=llvm]{assets/indvars.mlir}
\end{frame}

\begin{frame}[fragile]
    \frametitle{Final Result}

    \lstinputlisting[language=llvm]{assets/final.mlir}
\end{frame}

\begin{frame}[fragile]
    \frametitle{\texttt{affine} $\rightarrow$ \texttt{vector}}

    \begin{lstlisting}[language=llvm]
// send it to affine-supre-vertorize pass
// mlir-opt --affine-super-vectorize='virtual-vector-size=256 test-fastest-varying=0'
func.func @example_vector() {
    //... same as above
    // unimplement yet :(
    func.return
}
    \end{lstlisting}

    目前的 \texttt{affine} super-vectorize 只实现了少量 case !

\end{frame}

\begin{frame}[fragile]
    \frametitle{\texttt{affine} $\rightarrow$ \texttt{vector}}

    一个真实可用的例子:

    \lstinputlisting[language=llvm]{assets/affine-a+b.mlir}

    上面的 $C = A + B$ 可以被变为:

    \lstinputlisting[language=llvm]{assets/affine-vec.mlir}

\end{frame}

\begin{frame}
    \frametitle{Target Feature}

    一些体系结构支持 \texttt{vector} Dialect 之外的其他扩展,如何充分利用体系结构的能力?

    \vspace{1.5em}

    \begin{enumerate}
        \item 定义自己的 Dialect ,作为 \texttt{vector} Dialect 的补充
        \item Custom Vectorizer
        \item Lowering Path
    \end{enumerate}

\end{frame}


\begin{frame}
    \frametitle{Conclusion}

    对于某一个新体系结构,要想针对其实现自动向量化,可以的方案:

    \begin{enumerate}
        \item 观察 \texttt{vector} Dialect 有没有已有的抽象,将没有的抽象自定义 Dialect
        \item 自己实现自定义 Dialect 的自动向量化
        \item 自定义 Dialect 需要支持 Lowering 到 LLVM IR (可选 VP 或者 Vector)
        \item 实现 llc 后端,主要是 VP Intrinsics 和 Vector intrinsics 到自己的体系结构
    \end{enumerate}

\end{frame}

\section{RISC-V V Extension}

\begin{frame}
    \frametitle{RISC-V}

    RVV 被认为是面向``超级计算机''的扩展。RVV 支持多种现代向量指令集所具有的操作,从编译器设计角度带来了一些具体的挑战:

    \begin{table}
        \scriptsize
        \centering
        \caption{部分 RVV Feature 与对应编译器自动向量化实现情况}
        \begin{tabular}{cccc}
            \toprule
            RVV Feature         & 功能           & 传统 SIMD       & LLVM 实现             \\
            \midrule
            向量长度寄存器 \texttt{vl} & 控制要执行的向量长    & 无             & VP,向量化忽略            \\
            mask                & mask 掉不需要的   & 有,较少 (AVX512) & VP,向量化忽略            \\
            寄存器编组               & 向量寄存器可以组合或拆分 & 以不同指令区分不同向量长度 & \thead{基于 vscale 实现 \\ 向量化需要手动指定} \\
            \bottomrule
        \end{tabular}
    \end{table}

\end{frame}

\begin{frame}{Current RVV Vectorization Lowering Path}
    \begin{minipage}[t]{0.5\textwidth}
        \begin{table}
            \scriptsize
            \centering
            \caption{MLIR Lowering Path 实现情况}
            \begin{tabular}{ccc}
                \toprule
                序号                                                                                                                            & 实现情况 & 备注           \\
                \midrule
                (1)                                                                                                                           & 无    & 有参考资料        \\
                (2)\footnote{\tiny 可以在循环主体部分参考 SIMD 架构的实现, 循环尾的处理需要新的实现}                                                                      & 无    & 无(少)参考资料     \\
                (3) (6)                                                                                                                       & 成熟   &              \\
                (4)(5)\footnote{\tiny 基本无实现,需要对现有代码较大增补和验证,并且需要和上游交流,例如 Scalable 与非 Scalable 的混杂、原来的对 operation 的递归定义是否对 Scalable Vector 适用?} & 无    & Scalable 新加入 \\
                \bottomrule
            \end{tabular}
        \end{table}

        \scriptsize
        LLVM 后端有 vector intrinsic 和 vp intrinsic 两种后端可选。

        其中 vp 出现较晚,没有 vector 成熟,但基本能用。
    \end{minipage}%
    % \hfill%
    \begin{minipage}[t]{0.5\textwidth}
        \begin{figure}
            \centering
            \includegraphics[width=0.5\linewidth]{images/lowering-path-around-vector.png}
            \caption{Lowering path around vector dialect}
        \end{figure}
    \end{minipage}
\end{frame}

\end{document}
